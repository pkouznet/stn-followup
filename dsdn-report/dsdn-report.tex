\documentclass[11pt,pdftex,letter]{article}
%\documentclass[11pt]{llncs}
%\documentclass[11pt,pdftex]{article}
%\documentclass{sig-alternate-10pt}
%\usepackage{amsthm}
%\usepackage{algorithm}
%\usepackage[noend]{algorithmic}
\usepackage[lined,boxed,commentsnumbered]{algorithm2e}


\usepackage{amssymb}
\usepackage{comment}
\usepackage{amsmath}
%\usepackage{graphicx}
%\usepackage{color}
\usepackage{fullpage}
%\usepackage[pdftex]{graphicx}
%\DeclareGraphicsRule{*}{mps}{*}{<++>}

\ifx\pdftexversion\undefined
\usepackage[dvips]{graphicx}
\else
  \usepackage[pdftex]{graphicx}
  \DeclareGraphicsRule{*}{mps}{*}{}
\fi


\def\lf{\tiny}
\def\rrnnll{\setcounter{linenumber}{0}}
\def\nnll{\refstepcounter{linenumber}\lf\thelinenumber}
\newcounter{linenumber}

\usepackage[usenames,dvipsnames]{color}
%\usepackage[caption=true,font=footnotesize]{subfig}
\usepackage{subfigure}
%\usepackage{xspace} 	% Guesses whether a space is needed when invoked
\usepackage{listings}
\usepackage{url}
\usepackage{wrapfig}
\usepackage{cite}
%\usepackage{framed}
\usepackage{framed,color}
\definecolor{shadecolor}{rgb}{0.9,0.9,0.9}
%
\usepackage[boxed]{algorithm2e}
%\usepackage{comment}

%[[PKto change spacing
%\usepackage{titlesec}
%\titlespacing\section{0pt}{7pt}{6pt}
%]]

%\newtheorem{theorem}{Theorem}[section]
%\newtheorem{takeaway}[theorem]{Takeaway}
%\newtheorem{fact}[theorem]{Fact}


%\pdfpagewidth=8.5in
%\pdfpageheight=11in

% url.sty was written by Donald Arseneau. It provides better support for
% handling and breaking URLs. url.sty is already installed on most LaTeX
% systems. The latest version can be obtained at:
% http://www.ctan.org/tex-archive/macros/latex/contrib/misc/
% Read the url.sty source comments for usage information. Basically,
% \url{my_url_here}.

\newcommand{\CPO}{\textsc{FixTag}}
\newcommand{\DPO}{\textsc{ReuseTag}}
\newcommand{\PS}{\textsc{PS}}
\newcommand{\Bit}{\textsc{Bit}}

\newcommand{\gns}{Global Network State\xspace}
\newcommand{\GNS}{\textsc{GNS}\xspace}

%\newtheorem{theorem}{Theorem}[section]
%\newtheorem{lemma}{Lemma}[section]
%\newtheorem{claim}{Claim}[section]
%\newtheorem{scenarios}{Scenarios}[section]
%\newtheorem{observation}{Observation}
%\newtheorem{takeaway}{Takeaway}[section]
%\newtheorem{definition}{Definition}


%
%
% carry over Herald's group cool way of marking changes
\definecolor{heraldBlue}{rgb}{0.0,0.0,0.8}
\definecolor{heraldRed}{rgb}{0.8,0.0,0.0}
\definecolor{heraldGray}{rgb}{0.4,0.4,0.4}
\definecolor{heraldBlack}{rgb}{0.0,0.0,0.0} %removes comment color
\definecolor{heraldGreen}{rgb}{0.0,0.4,0.0} %removes comment color
\def\r#1{\textcolor{heraldBlue}{\em #1}}
\def\q#1{\textcolor{heraldRed}{\em #1}}
\def\d#1{\textcolor{heraldBlue}{#1}}
\def\R#1{\textcolor{heraldBlue}{#1}}
\def\D#1{\textcolor{heraldBlue}{#1}}
%
%\DeclareMathOperator{\respc}{res}

\newcommand{\len}{\text{d}}

\newcommand{\nodes}{\mathcal{N}}
\newcommand{\links}{\mathcal{E}}
\newcommand{\epoints}{\Pi}
\newcommand{\epoint}{\pi}
\newcommand{\legsw}{\mathcal{L}}
\newcommand{\sdnsw}{\mathcal{S}}
\newcommand{\sw}{\legsw\sqcup\sdnsw}
\newcommand{\fset}{FS}

\newcommand{\Cost}{\gamma}

\newcommand{\ft}{ft}

\newcommand{\eepath}{p}
\newcommand{\link}{e}

\newcommand{\dom}{\textit{dom}}
\newcommand{\pr}{\textit{pr}}
\newcommand{\CPOs}{\textit{paths}}
\newcommand{\seq}{\textit{seq}}
\newcommand{\cur}{\textit{cur}}
\newcommand{\Tag}{\textit{tag}}


\newcommand{\ports}{\Pi}
\newcommand{\seport}{\pi}
\newcommand{\inports}{\overrightarrow{\ports}}
\newcommand{\swports}{\overleftrightarrow{\ports}}
\newcommand{\seinports}{\epoints^{\bullet}}
\newcommand{\seinportsone}{\seinports_1}
\newcommand{\seinportstwo}{\seinports_2}
\newcommand{\seinportsthree}{\seinports_3}
\newcommand{\neinports}{\epoints^{\circ}}

\newcommand{\readt}{\texttt{r}}
\newcommand{\writet}{\texttt{w}}
\newcommand{\op}{\texttt{op}}


\newcommand{\cellblocks}{CB}
\newcommand{\cellblock}{c}
\newcommand{\frontier}{\mathcal{F}}
\newcommand{\MIP}{\textsc{Opt}}
\newcommand{\smartparagraph}[1]{\noindent{\bf #1}\ }
\newcommand{\eg}{{\it e.g.}}
\newcommand{\ie}{{\it i.e.}}
\newcommand{\etc}{{\it etc.}}
\newcommand{\etal}{{\it et al.}\xspace}
\newcommand{\id}{{\it id}}

\def\TR{0}
\def\NOTES{1}
\def\SAVESPACE{1}
\def\SHOWAUTHORS{1}
\def\SHOWGIT{0}
% variables may contain the above definitions to control build options at compile-time
%\input{variables}

\if \SAVESPACE 1
%\usepackage{setspace}
%\usepackage{titlesec}
%\titlespacing\section{0pt}{7pt}{6pt}
\usepackage{titlesec}
\titlespacing\section{0pt}{7pt}{6pt}

\fi

\if \NOTES 1
\newcommand{\mcnote}[1]{\textcolor{heraldBlue}{\small \bf [MC: #1]}}
\newcommand{\dlnote}[1]{\textcolor{heraldBlue}{\small \bf [DL: #1]}}
\newcommand{\ssnote}[1]{\textcolor{heraldBlue}{\small \bf [SS: #1]}}
\newcommand{\pknote}[1]{\textcolor{heraldBlue}{\small \bf [PK: #1]}}
%\newcommand{\problem}[1]{\textcolor{heraldRed}{\small \bf [PROBLEM: #1]}}
\else
\newcommand{\mcnote}[1]{}
\newcommand{\dlnote}[1]{}
\newcommand{\ssnote}[1]{}
\newcommand{\pknote}[1]{}
\newcommand{\problem}[1]{}
\fi

%Petr's definitions
\newcommand{\ack}{\textit{ack}}
\newcommand{\nack}{\textit{nack}}
\newcommand{\rmw}{\textit{rmw}}
\newcommand{\State}{\mathit{States}}
\newcommand{\ignore}[1]{}
\newcommand{\Pb}{CPC}
\newcommand{\E}{\mathcal E}
%

%[[PK environments for article style
\newtheorem{theorem}{Theorem}
\newtheorem{conjecture}[theorem]{Conjecture}
\newtheorem{corollary}[theorem]{Corollary}
\newtheorem{definition}[theorem]{Definition}
\newtheorem{lemma}[theorem]{Lemma}
\newtheorem{observation}[theorem]{Observation}
\newenvironment{proof}[1][Proof]{\noindent\textbf{#1.} }{\hfill $\Box$\\[2mm]}
\newenvironment{proofsketch}[1][Proof sketch]{\noindent\textbf{#1.} }{\hfill $\Box$\\[2mm]}
%]]

\begin{document}
\sloppy

%\title{Distributed Network Programming}
%\title{The Distributed SDN Update Problem:\\Towards Software Transactional Networks}

%\title{On Consistent Updates in Software Defined Networks under Unreliable Control}
%\title{The Case for Reliable Software Transactional Networking}

%\title{A Distributed SDN Control Plane for Concurrent Policy Updates}

\title{Distributed Software-Defined Networking:\\ ACM PODC 2014 Workshop }



\author{
Petr Kuznetsov$^{1}$ \quad Stefan Schmid$^{2}$\\
\\
       $^{1}$ T\'el\'ecom ParisTech, 46 Rue Barrault, 75013 Paris, France\\
        petr.kuznetsov@telecom-paristech.fr\\
\\
        $^{2}$ TU Berlin \& T-Labs, Ernst-Reuter Platz 7, 10587 Berlin, Germany\\
	    stefan.schmid@tu-berlin.de}
%	}

%\institute{}

\date{}


\maketitle


\thispagestyle{empty}

%\if \SAVESPACE 1
%\setlength{\floatsep}{3pt}
%\setlength{\textfloatsep}{3pt}
%\setlength{\dbltextfloatsep}{3pt}
%\setlength{\intextsep}{3pt}
%\setlength{\abovecaptionskip}{3pt}
%\fi

% A category with the (minimum) three required fields
%\category{C.2.1}{Network Architecture and Design}{Centralized Networks}
%\category{C.2.4}{Distributed Systems}{Network Operating Systems}
%\terms{Measurement, Performance}
%\keywords{}



\begin{abstract}

The first workshop on Distributed Software-Defined Networking took
place in Paris, France, on the 15th of July, just before 
the 33rd ACM Symposium on Principles of Distributed Computing. 
The workshop intended to be a forum to discuss new algorithmic and
distributed computing challenges offered by the emerging field of
Software Defined Networking (SDN). SDN abstracts the cumbersome task 
of computer network management to a distinct control plane with a
standard programming interface. 
In th worshop, we invited invited researchers in the fields of
distributed computing and
networking in order to understand whether distributed implementations of the
SDN control 
plane give rise to new and interesting research questions, where the
PODC community may use its expertise.
\end{abstract}


\section{Distributed SDNs: New? Interesting?}

The emerging paradigm of Software-Defined Networking (SDN) promises to
simplify network management and enable building networks
that meet specific, end-to-end requirements.
%SDN centralizes on
In SDN, the \emph{control plane}  (a collection of network-attached servers)
maintains control over the so-called \emph{data plane} (the
packet-forwarding functionality implemented on switching
hardware). Control applications operate on a global, logically-centralized network view,
which introduces opportunities for network-wide management and optimization.
This view enables simplified programming models to define a high-level
network policy that the data plane must respect.

There is no doubt that implementing a ``logically centralized'' SDN
control on a single centralized controller is an overkill in terms of
availability, responsiveness and throughput.      
As a recent survey on advances in SDN
research~\cite{sdn-survey} puts it:  ``As
the initial concerns with SDN scalability were addressed ...
– in particular the myth that logical centralization implied
a physically centralized controller, ... SDN ideas have matured and evolved from
an academic exercise to a commercial success.'' 

But does the perspective of distributed SDN evoke new and interesting
research questions for the distributed computing community? Or the SDN
arhitects can simply resort to existing solutions designed for other
types of systems?

In the first edition of the workshop devoted to \emph{distributed}
aspects of software-defined networking (affiliated with
PODC 2014), we tried to address this question. 

\section{Program}


\section{Summary}


\bibliographystyle{abbrv}
\bibliography{references}  % main.bib is the name of the Bibliography in this case

\end{document}
