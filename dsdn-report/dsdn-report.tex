\documentclass[11pt,pdftex,letter]{article}
%\documentclass[11pt]{llncs}
%\documentclass[11pt,pdftex]{article}
%\documentclass{sig-alternate-10pt}
%\usepackage{amsthm}
%\usepackage{algorithm}
%\usepackage[noend]{algorithmic}
\usepackage[lined,boxed,commentsnumbered]{algorithm2e}


\usepackage{amssymb}
\usepackage{comment}
\usepackage{amsmath}
%\usepackage{graphicx}
%\usepackage{color}
\usepackage{fullpage}
%\usepackage[pdftex]{graphicx}
%\DeclareGraphicsRule{*}{mps}{*}{<++>}

\ifx\pdftexversion\undefined
\usepackage[dvips]{graphicx}
\else
  \usepackage[pdftex]{graphicx}
  \DeclareGraphicsRule{*}{mps}{*}{}
\fi


\def\lf{\tiny}
\def\rrnnll{\setcounter{linenumber}{0}}
\def\nnll{\refstepcounter{linenumber}\lf\thelinenumber}
\newcounter{linenumber}

\usepackage[usenames,dvipsnames]{color}
%\usepackage[caption=true,font=footnotesize]{subfig}
\usepackage{subfigure}
%\usepackage{xspace} 	% Guesses whether a space is needed when invoked
\usepackage{listings}
\usepackage{url}
\usepackage{wrapfig}
\usepackage{cite}
%\usepackage{framed}
\usepackage{framed,color}
\definecolor{shadecolor}{rgb}{0.9,0.9,0.9}
%
\usepackage[boxed]{algorithm2e}
%\usepackage{comment}

%[[PKto change spacing
%\usepackage{titlesec}
%\titlespacing\section{0pt}{7pt}{6pt}
%]]

%\newtheorem{theorem}{Theorem}[section]
%\newtheorem{takeaway}[theorem]{Takeaway}
%\newtheorem{fact}[theorem]{Fact}


%\pdfpagewidth=8.5in
%\pdfpageheight=11in

% url.sty was written by Donald Arseneau. It provides better support for
% handling and breaking URLs. url.sty is already installed on most LaTeX
% systems. The latest version can be obtained at:
% http://www.ctan.org/tex-archive/macros/latex/contrib/misc/
% Read the url.sty source comments for usage information. Basically,
% \url{my_url_here}.

\newcommand{\CPO}{\textsc{FixTag}}
\newcommand{\DPO}{\textsc{ReuseTag}}
\newcommand{\PS}{\textsc{PS}}
\newcommand{\Bit}{\textsc{Bit}}

\newcommand{\gns}{Global Network State\xspace}
\newcommand{\GNS}{\textsc{GNS}\xspace}

%\newtheorem{theorem}{Theorem}[section]
%\newtheorem{lemma}{Lemma}[section]
%\newtheorem{claim}{Claim}[section]
%\newtheorem{scenarios}{Scenarios}[section]
%\newtheorem{observation}{Observation}
%\newtheorem{takeaway}{Takeaway}[section]
%\newtheorem{definition}{Definition}


%
%
% carry over Herald's group cool way of marking changes
\definecolor{heraldBlue}{rgb}{0.0,0.0,0.8}
\definecolor{heraldRed}{rgb}{0.8,0.0,0.0}
\definecolor{heraldGray}{rgb}{0.4,0.4,0.4}
\definecolor{heraldBlack}{rgb}{0.0,0.0,0.0} %removes comment color
\definecolor{heraldGreen}{rgb}{0.0,0.4,0.0} %removes comment color
\def\r#1{\textcolor{heraldBlue}{\em #1}}
\def\q#1{\textcolor{heraldRed}{\em #1}}
\def\d#1{\textcolor{heraldBlue}{#1}}
\def\R#1{\textcolor{heraldBlue}{#1}}
\def\D#1{\textcolor{heraldBlue}{#1}}
%
%\DeclareMathOperator{\respc}{res}

\newcommand{\len}{\text{d}}

\newcommand{\nodes}{\mathcal{N}}
\newcommand{\links}{\mathcal{E}}
\newcommand{\epoints}{\Pi}
\newcommand{\epoint}{\pi}
\newcommand{\legsw}{\mathcal{L}}
\newcommand{\sdnsw}{\mathcal{S}}
\newcommand{\sw}{\legsw\sqcup\sdnsw}
\newcommand{\fset}{FS}

\newcommand{\Cost}{\gamma}

\newcommand{\ft}{ft}

\newcommand{\eepath}{p}
\newcommand{\link}{e}

\newcommand{\dom}{\textit{dom}}
\newcommand{\pr}{\textit{pr}}
\newcommand{\CPOs}{\textit{paths}}
\newcommand{\seq}{\textit{seq}}
\newcommand{\cur}{\textit{cur}}
\newcommand{\Tag}{\textit{tag}}


\newcommand{\ports}{\Pi}
\newcommand{\seport}{\pi}
\newcommand{\inports}{\overrightarrow{\ports}}
\newcommand{\swports}{\overleftrightarrow{\ports}}
\newcommand{\seinports}{\epoints^{\bullet}}
\newcommand{\seinportsone}{\seinports_1}
\newcommand{\seinportstwo}{\seinports_2}
\newcommand{\seinportsthree}{\seinports_3}
\newcommand{\neinports}{\epoints^{\circ}}

\newcommand{\readt}{\texttt{r}}
\newcommand{\writet}{\texttt{w}}
\newcommand{\op}{\texttt{op}}


\newcommand{\cellblocks}{CB}
\newcommand{\cellblock}{c}
\newcommand{\frontier}{\mathcal{F}}
\newcommand{\MIP}{\textsc{Opt}}
\newcommand{\smartparagraph}[1]{\noindent{\bf #1}\ }
\newcommand{\eg}{{\it e.g.}}
\newcommand{\ie}{{\it i.e.}}
\newcommand{\etc}{{\it etc.}}
\newcommand{\etal}{{\it et al.}\xspace}
\newcommand{\id}{{\it id}}

\def\TR{0}
\def\NOTES{1}
\def\SAVESPACE{1}
\def\SHOWAUTHORS{1}
\def\SHOWGIT{0}
% variables may contain the above definitions to control build options at compile-time
%\input{variables}

\if \SAVESPACE 1
%\usepackage{setspace}
%\usepackage{titlesec}
%\titlespacing\section{0pt}{7pt}{6pt}
\usepackage{titlesec}
\titlespacing\section{0pt}{7pt}{6pt}

\fi

\if \NOTES 1
\newcommand{\mcnote}[1]{\textcolor{heraldBlue}{\small \bf [MC: #1]}}
\newcommand{\dlnote}[1]{\textcolor{heraldBlue}{\small \bf [DL: #1]}}
\newcommand{\ssnote}[1]{\textcolor{heraldBlue}{\small \bf [SS: #1]}}
\newcommand{\pknote}[1]{\textcolor{heraldBlue}{\small \bf [PK: #1]}}
%\newcommand{\problem}[1]{\textcolor{heraldRed}{\small \bf [PROBLEM: #1]}}
\else
\newcommand{\mcnote}[1]{}
\newcommand{\dlnote}[1]{}
\newcommand{\ssnote}[1]{}
\newcommand{\pknote}[1]{}
\newcommand{\problem}[1]{}
\fi

%Petr's definitions
\newcommand{\ack}{\textit{ack}}
\newcommand{\nack}{\textit{nack}}
\newcommand{\rmw}{\textit{rmw}}
\newcommand{\State}{\mathit{States}}
\newcommand{\ignore}[1]{}
\newcommand{\Pb}{CPC}
\newcommand{\E}{\mathcal E}
%

%[[PK environments for article style
\newtheorem{theorem}{Theorem}
\newtheorem{conjecture}[theorem]{Conjecture}
\newtheorem{corollary}[theorem]{Corollary}
\newtheorem{definition}[theorem]{Definition}
\newtheorem{lemma}[theorem]{Lemma}
\newtheorem{observation}[theorem]{Observation}
\newenvironment{proof}[1][Proof]{\noindent\textbf{#1.} }{\hfill $\Box$\\[2mm]}
\newenvironment{proofsketch}[1][Proof sketch]{\noindent\textbf{#1.} }{\hfill $\Box$\\[2mm]}
%]]

\begin{document}
\sloppy

%\title{Distributed Network Programming}
%\title{The Distributed SDN Update Problem:\\Towards Software Transactional Networks}

%\title{On Consistent Updates in Software Defined Networks under Unreliable Control}
%\title{The Case for Reliable Software Transactional Networking}

%\title{A Distributed SDN Control Plane for Concurrent Policy Updates}

\title{Distributed Software-Defined Networking:\\ The ACM PODC 2014 Workshop \textbf{DSDN}}



\author{
Petr Kuznetsov$^{1}$ \quad Stefan Schmid$^{2}$\\
\\
       $^{1}$ T\'el\'ecom ParisTech\\
%, 46 Rue Barrault, 75013 Paris, France\\
        petr.kuznetsov@telecom-paristech.fr\\
\\
        $^{2}$ TU Berlin \& T-Labs \\ %Ernst-Reuter Platz 7, 10587 Berlin, Germany\\
	    stefan.schmid@tu-berlin.de}
%	}

%\institute{}

\date{}


\maketitle


\thispagestyle{empty}

%\if \SAVESPACE 1
%\setlength{\floatsep}{3pt}
%\setlength{\textfloatsep}{3pt}
%\setlength{\dbltextfloatsep}{3pt}
%\setlength{\intextsep}{3pt}
%\setlength{\abovecaptionskip}{3pt}
%\fi

% A category with the (minimum) three required fields
%\category{C.2.1}{Network Architecture and Design}{Centralized Networks}
%\category{C.2.4}{Distributed Systems}{Network Operating Systems}
%\terms{Measurement, Performance}
%\keywords{}



\begin{abstract}
The workshop on Distributed Software-Defined Networking, DSDN, took
place in Paris, France, on the 15th of July, just before
the 33rd ACM Symposium on Principles of Distributed Computing.
The workshop intended to be a forum to discuss new algorithmic and
distributed computing challenges offered by the emerging field of
Software Defined Networking (SDN).
The workshop consisted of invited as well as peer-reviewed presentations, both
from researchers in the field of
distributed computing and in the field of networking. 
\end{abstract}


\section{SDN: Networking Is Cool Again!}

Computer networking currently goes through a transition phase,
and the paradigm of Software-Defined Networking (SDN) 
is discussed intensively, both in the industry and in the academia. In
a nutshell, the paradigm out-sources and consolidates the control over a network
to a logically centralized \emph{software control plane}.
This separation, and the introduction of ``programmability'',
allows to adapt and innovate the network control plane more quickly
and independently of the data plane.
The resulting flexibilities open interesting new opportunities:
networking is considered cool again.

\section{SDN meets PODC}

At the heart of SDN lies the idea to design and operate 
the network control logic 
on a centralized network view.
However, inevitably, this view is only \emph{logically centralized}~\cite{onix}: 
in order to avoid a single point of failure and ensure scalability and efficiency,  
the control plane state must
be physically distributed. 

The design of a distributed control plane is only one example where we feel that the
PODC community could contribute to relevant networking problems today.
Accordingly, the goal of the DSDN workshop was to bring together networking researchers with the PODC
community, to discuss current trends in networking, and to identify interesting articulation
points between the two communities.

The workshop took
place in Paris on the sunny day of July 15th, just before
PODC, concurrently with four other PODC workshops.

\section{Program}

Below we give a short summary of the talks given in our
workshop. Abstracts and slides can be found at \url{http://www.podc.org/podc2014/dsdn14/}.
%Unfortunately, one of the speakers could not present his
%work due to visa issues.

\subsection{Foundations of SDN}

The opening keynote was given by Prof Nate Foster (\emph{Cornell University}),
one of the leading figures in SDN and whose research
is situated at the intersection of programming languages, networks, and security. 

Nate Foster gave an overview of the motivation for and foundations of SDN, and
emphasized the importance that programming languages and formal methods 
play in software-defined networks: Only by a careful engineering of the right programming abstractions,
an effective reasoning about network behavior becomes possible. The ability to
formally reason about an SDN system is important, especially in the light
of today trend towards public cloud computing, where a network misconfiguration may
leak confidential information to other tenants.

Nate Foster presented his vision of the Machine, Language, and Runtime Models for SDN.
In particular, he described the design of a machine-verified SDN controller,
which is based on a detailed operational OpenFlow model 
and which is formalized in Coq.~\cite{machine-verified} 
This operational OpenFlow model can also be used 
to develop a verified compiler and run-time system
for a more high-level network programming language such as NetKAT.~\cite{netkat} 


\subsection{Consistent Range Classification with OpenFlow}

Yehuda Afek (TAU) presented his recent work with Anat Bremler-Barr and
Liron Schiff~\cite{AfekBS14} on consistently classifying ranges 
in SDN networks with multiple entrance, i.e., in scenarios where the flow changes the entrance point
to the network. 
Their range classification scheme only requires three entries per range, 
and supports atomic updates across multiple switches. This makes the scheme attractive,
e.g., in load-balancing and NFV applications. 

We discuss issues in the ranges based flow classification in a dynamic SDN based network. Such issues arise in load-balancing and security based applications in a multi entrance network, where flows may dynamically change their point of entrance. We present a new consistent flow management scheme in an Openflow based SDN network. that addresses these issues. Our first step is a new efficient mutli-range classification scheme which uses only 3 entries per range instead of w per range as in the best existing classifiers, where w is the field size in bits. Building on the ranges classification we show how to update ranges across multiple switches in an atomic manner � allowing to update the set of ranges and their associated actions while packets are classified and the network is changing. Finally, using the two schemes above, we present an architecture suitable for several applications such as load-balancing (which we describe in detail), and NFV, to manage multi-entrance consistency � ensuring that flows are handled by the same policy even when they change the entrance point to the network. Future extensions will be discussed. Joint work with Anat Bremler-Barr and Liron Schiff.


\subsection{SDN-Based Private Interconnection}

Shlomi Dolev1 Shimrit Tzur-David

Cloud computing is one of the fastest growing opportunities for enterprises and service providers. Enterprises use the Infrastructureas-
a-Service (IaaS) model to build private and public clouds that reduce operating and capital expenses and increase the agility
and reliability of their critical information systems. In order to fulfil these needs, service providers build public clouds to offer
on-demand, secure, multi-tenant IT infrastructure to potential costumers that now can use cloud services using a public cloud
infrastructure.
The Open Networking Foundation (ONF) report on the infrastructure between datacenters [?] states that fast-changing
and demanding enterprise and carrier business requirements force changes in network architecture. However, these changes
were focused on datacenter server and storage virtualization, while the underpinning network architectures have stagnated with
respect to both scalability and manageability.
Due to the growth of businesses and the advent of Big Data, the private clouds are augmented with external resources known
as public clouds. The use of public cloud requires efficient and well performed connectivity between private and public clouds.
This resulting �hybrid� cloud should provide transfer and sharing of data. Furthermore, this transfer has to be private.
SDN enables more deterministic, more scalable, more manageable and as we present in this paper, also private virtual
networks between the local datacenters that reside in the private cloud, to the public resources in the public cloud. These virtual
networks are called the hybrid cloud.
In this work we present a private hybrid cloud in which all the information that pass across the cloud is information theoretic
secured. I.e., unless there is a coalition of several routers in the cloud, the information cannot be revealed. This is done by using
secret sharing scheme together with SDN to ensure privacy. 
%Encryption with (n;k) secret sharing scheme (n  k) is done by
creating n shares from the data such that only by having at least k shares, the data can be decrypted. In the cloud notations,
assume that the data has to be sent from the private datacenter. The source in the private datacenter creates n shares from the
data and sends them to the destination at the public cloud through the hybrid cloud. The SDN controller manages the routes of
these shares such that no router sees k or more shares. This way, we ensure that only the destination at the public cloud that gets
%all the shares, can decrypt the data, resulting in a private channel in the hybrid cloud. When n > k, we allow n..k shares to get
lost, due to congestion or even by malicious routers.
Our Contribution. The contributions of our paper are as follows; To the best of our knowledge, we are the first to use secret
sharing for a unicast communication over SDN architecture. We show that secret sharing can be very useful to achieve private
channel when two parties communicate over a multipath network. Whereas many papers have some contribution on the area
of public cloud security, i.e., securing data in the cloud, to the best of our knowledge, we are the first to target the problem
of theoretical secured channel to the public cloud. We show that even if there is a probability to the existence of coalition of
several routers in the hybrid cloud, we can still bound the probability for privacy violation.
2 Gaining


\subsection{Software Transactional Networking: A Robust and
  Distributed SDN Control Plane}

Marco Canini (UCL) presented the concept of software transactional
networking (STN), a control-plane abstraction used for consistent
composition of concurrent policies~\cite{stn,tr-stn}. The abstraction assumes a set of
control applications that concurrently apply modifications (or
updates) of the
\emph{network policy}, i.e., the set of rules that stipulate how the
traffic should be processed at the data plane.
Given that the policy updates coming from different control applications may
conflict with each other, the STN framework offers to the applications
a \emph{transactional} interface with \emph{all-or-nothing} semantics.
The appealing difference with classical transactional systems is that
here we have to make sure that some kinds of transactions, namely
\emph{data-plane traffic} traces have to be processed in a manner that
is transparent to network-configuration changes.
The talk hinted on the formal definition of the abstraction of consistent policy
composition (CPC), and sketched designs of its resilient implementations.

\subsection{Declarative, Distributed Configuration}

Can the challenges motivating the use of SDN be addressed using
existing hardware and protocols. Sanjai Narain (Applied Communication Sciences) summarized the
experience collected at his company in using the Assured and Dynamic
Configuration(ADC) system. In their approach, network functionality is
expressed as a set of constraints on configuration variables. SAT or
SMT solvers are used to convert these constraints into values of configuration variables.
Of course, in specific scenarios, proprietary solutions may be more
efficient and  easier to deploy than generic ones (e.g., based on the
SDN framework), and Sanjai's talk questioned the very
motivation behind migrating to SDN.

\subsection{Managing the Network with Merlin}

This talk presents the Merlin network management framework. With Merlin, administrators express network policy using programs in a declarative language based on logical predicates and regular expressions. The Merlin compiler automatically partitions these programs into components that can be placed on a variety of devices.
It uses a constraint solver to allocate resources such as paths and bandwidth. To ease the administration of federated networks, Merlin provides mechanisms for delegating management of sub-policies to tenants, along with tools for verifying that delegated sub-policies do not violate global constraints. Overall, Merlin greatly simplifies the task of network administration.


Merlin:
A Language for
Provisioning
Network Resources

How to Program The Network?
2
Existing SDN languages focus mostly
on packet forwarding
Ignore other vital network features like
bandwidth, packet processing, etc.
Network orchestration frameworks
expose extremely simple APIs (if at all)

Merlin Approach
3
Specify
Compile
Specify global network policy in a
high-level declarative language.
Map to a constraint problem.
Provision network, select paths,
and decide function placement.
Generate device-specific code and
configuration to enforce policy.



Robert Soule Managing the Network with Merlin
Abstract: This talk presents the Merlin network management framework. With Merlin, administrators express network policy using programs in a declarative language based on logical predicates and regular expressions. The Merlin compiler automatically partitions these programs into components that can be placed on a variety of devices.
It uses a constraint solver to allocate resources such as paths and bandwidth. To ease the administration of federated networks, Merlin provides mechanisms for delegating management of sub-policies to tenants, along with tools for verifying that delegated sub-policies do not violate global constraints. Overall, Merlin greatly simplifies the task of network administration.

\subsection{Managing Dynamic Networks: Distributed or Centralized
  Control}

What if a node or edge of a network fails? What if traffic between two nodes grows or shrinks? Clearly this is a case for distributed algorithms! After all, the Internet and its protocols are decentralized by design. Thanks to redundancy and distributed communications, Paul Baran suggested that the Internet could operate even if many of its links and nodes had been destroyed by a �nuclear attack�. However, it turns out that distributed control leaves quite some efficiency on the table. Recently several companies started to organize their enterprise networks in a centralized fashion, separating data and control planes by means of software-defined networks (SDNs). So we might soon see a world where a (possibly fault-tolerant but nevertheless) centralized controller is making all routing and transport decisions, using SDN-switches to implement these decisions. Is this the death of distributed network algorithms? In my talk I will discuss why this is not necessarily true, that is, why even central control may learn from distributed computing when dealing with network dynamics.

In his somewhat provocative talk, Roger Wattenhofer (ETHZ) considers
the following question. If we take the extreme case of a network
managed (e.g., using the SDN approach) by a (fault-tolerant,
performant, etc.) central controller, would this imply
that distributed algorithms are no longer needed. The answer (not very
surprisingly) is no: we still have to deal with the problem of
dynamicity and failures on the data plane, as well as the fact that
the data plane is inherently geographically distributed and cannot be
manipulated in the atomic manner. In short, we still have to deal with
\emph{consistency} of network control, and here we can benefit from
distributed computing which is essentially all about consistency.
In his talk, Roger overviewed several natural network consistency
criteria (such as \emph{loop-freedom} or \emph{per-packet
  consistency}) and sketched several impossibility results and
complexity bounds of achieving these criteria in a few different
network models.


\section{Summary and Outlook}

The DSDN workshop gathered a good number of participants,
which
fluctuated around 20-40 people, depending on the time-of-day.
For several DSDN participants, it was the first time
to participate in a PODC event, and they used the opportunity
to stay throughout the entire conference.
We are very happy with this outcome, and consider the
workshop a success.

Interestingly, among the 400$+$ articles cited in a recent SDN survey~\cite{sdn-survey}, 
we do not find
a single paper that appeared in PODC or DISC.
Accordingly, we hope that the DSDN workshop could
provide the networking and systems community with a glimpse in to the
world of PODC. We also hope that the workshop could help to demystify
SDN, and inspired the PODC community
to tackle some of the fundamental distributed and
optimization problems offered by SDN.


\bibliographystyle{abbrv}
\bibliography{references}  % main.bib is the name of the Bibliography in this case

\end{document}
