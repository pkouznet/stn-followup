\documentclass[11pt,pdftex,letter]{article}
%\documentclass[11pt]{llncs}
%\documentclass[11pt,pdftex]{article}
%\documentclass{sig-alternate-10pt}
%\usepackage{amsthm}
%\usepackage{algorithm}
%\usepackage[noend]{algorithmic}
\usepackage[lined,boxed,commentsnumbered]{algorithm2e}


\usepackage{amssymb}
\usepackage{comment}
\usepackage{amsmath}
%\usepackage{graphicx}
%\usepackage{color}
\usepackage{fullpage}
%\usepackage[pdftex]{graphicx}
%\DeclareGraphicsRule{*}{mps}{*}{<++>}

\ifx\pdftexversion\undefined
\usepackage[dvips]{graphicx}
\else
  \usepackage[pdftex]{graphicx}
  \DeclareGraphicsRule{*}{mps}{*}{}
\fi


\def\lf{\tiny}
\def\rrnnll{\setcounter{linenumber}{0}}
\def\nnll{\refstepcounter{linenumber}\lf\thelinenumber}
\newcounter{linenumber}

\usepackage[usenames,dvipsnames]{color}
%\usepackage[caption=true,font=footnotesize]{subfig}
\usepackage{subfigure}
%\usepackage{xspace} 	% Guesses whether a space is needed when invoked
\usepackage{listings}
\usepackage{url}
\usepackage{wrapfig}
\usepackage{cite}
%\usepackage{framed}
\usepackage{framed,color}
\definecolor{shadecolor}{rgb}{0.9,0.9,0.9}
%
\usepackage[boxed]{algorithm2e}
%\usepackage{comment}

%[[PKto change spacing
%\usepackage{titlesec}
%\titlespacing\section{0pt}{7pt}{6pt}
%]]

%\newtheorem{theorem}{Theorem}[section]
%\newtheorem{takeaway}[theorem]{Takeaway}
%\newtheorem{fact}[theorem]{Fact}


%\pdfpagewidth=8.5in
%\pdfpageheight=11in

% url.sty was written by Donald Arseneau. It provides better support for
% handling and breaking URLs. url.sty is already installed on most LaTeX
% systems. The latest version can be obtained at:
% http://www.ctan.org/tex-archive/macros/latex/contrib/misc/
% Read the url.sty source comments for usage information. Basically,
% \url{my_url_here}.

\newcommand{\CPO}{\textsc{FixTag}}
\newcommand{\DPO}{\textsc{ReuseTag}}
\newcommand{\PS}{\textsc{PS}}
\newcommand{\Bit}{\textsc{Bit}}

\newcommand{\gns}{Global Network State\xspace}
\newcommand{\GNS}{\textsc{GNS}\xspace}

%\newtheorem{theorem}{Theorem}[section]
%\newtheorem{lemma}{Lemma}[section]
%\newtheorem{claim}{Claim}[section]
%\newtheorem{scenarios}{Scenarios}[section]
%\newtheorem{observation}{Observation}
%\newtheorem{takeaway}{Takeaway}[section]
%\newtheorem{definition}{Definition}


%
%
% carry over Herald's group cool way of marking changes
\definecolor{heraldBlue}{rgb}{0.0,0.0,0.8}
\definecolor{heraldRed}{rgb}{0.8,0.0,0.0}
\definecolor{heraldGray}{rgb}{0.4,0.4,0.4}
\definecolor{heraldBlack}{rgb}{0.0,0.0,0.0} %removes comment color
\definecolor{heraldGreen}{rgb}{0.0,0.4,0.0} %removes comment color
\def\r#1{\textcolor{heraldBlue}{\em #1}}
\def\q#1{\textcolor{heraldRed}{\em #1}}
\def\d#1{\textcolor{heraldBlue}{#1}}
\def\R#1{\textcolor{heraldBlue}{#1}}
\def\D#1{\textcolor{heraldBlue}{#1}}
%
%\DeclareMathOperator{\respc}{res}

\newcommand{\len}{\text{d}}

\newcommand{\nodes}{\mathcal{N}}
\newcommand{\links}{\mathcal{E}}
\newcommand{\epoints}{\Pi}
\newcommand{\epoint}{\pi}
\newcommand{\legsw}{\mathcal{L}}
\newcommand{\sdnsw}{\mathcal{S}}
\newcommand{\sw}{\legsw\sqcup\sdnsw}
\newcommand{\fset}{FS}

\newcommand{\Cost}{\gamma}

\newcommand{\ft}{ft}

\newcommand{\eepath}{p}
\newcommand{\link}{e}

\newcommand{\dom}{\textit{dom}}
\newcommand{\pr}{\textit{pr}}
\newcommand{\CPOs}{\textit{paths}}
\newcommand{\seq}{\textit{seq}}
\newcommand{\cur}{\textit{cur}}
\newcommand{\Tag}{\textit{tag}}


\newcommand{\ports}{\Pi}
\newcommand{\seport}{\pi}
\newcommand{\inports}{\overrightarrow{\ports}}
\newcommand{\swports}{\overleftrightarrow{\ports}}
\newcommand{\seinports}{\epoints^{\bullet}}
\newcommand{\seinportsone}{\seinports_1}
\newcommand{\seinportstwo}{\seinports_2}
\newcommand{\seinportsthree}{\seinports_3}
\newcommand{\neinports}{\epoints^{\circ}}

\newcommand{\readt}{\texttt{r}}
\newcommand{\writet}{\texttt{w}}
\newcommand{\op}{\texttt{op}}


\newcommand{\cellblocks}{CB}
\newcommand{\cellblock}{c}
\newcommand{\frontier}{\mathcal{F}}
\newcommand{\MIP}{\textsc{Opt}}
\newcommand{\smartparagraph}[1]{\noindent{\bf #1}\ }
\newcommand{\eg}{{\it e.g.}}
\newcommand{\ie}{{\it i.e.}}
\newcommand{\etc}{{\it etc.}}
\newcommand{\etal}{{\it et al.}\xspace}
\newcommand{\id}{{\it id}}

\def\TR{0}
\def\NOTES{1}
\def\SAVESPACE{1}
\def\SHOWAUTHORS{1}
\def\SHOWGIT{0}
% variables may contain the above definitions to control build options at compile-time
%\input{variables}

\if \SAVESPACE 1
%\usepackage{setspace}
%\usepackage{titlesec}
%\titlespacing\section{0pt}{7pt}{6pt}
\usepackage{titlesec}
\titlespacing\section{0pt}{7pt}{6pt}

\fi

\if \NOTES 1
\newcommand{\mcnote}[1]{\textcolor{heraldBlue}{\small \bf [MC: #1]}}
\newcommand{\dlnote}[1]{\textcolor{heraldBlue}{\small \bf [DL: #1]}}
\newcommand{\ssnote}[1]{\textcolor{heraldBlue}{\small \bf [SS: #1]}}
\newcommand{\pknote}[1]{\textcolor{heraldBlue}{\small \bf [PK: #1]}}
%\newcommand{\problem}[1]{\textcolor{heraldRed}{\small \bf [PROBLEM: #1]}}
\else
\newcommand{\mcnote}[1]{}
\newcommand{\dlnote}[1]{}
\newcommand{\ssnote}[1]{}
\newcommand{\pknote}[1]{}
\newcommand{\problem}[1]{}
\fi

%Petr's definitions
\newcommand{\ack}{\textit{ack}}
\newcommand{\nack}{\textit{nack}}
\newcommand{\rmw}{\textit{rmw}}
\newcommand{\State}{\mathit{States}}
\newcommand{\ignore}[1]{}
\newcommand{\Pb}{CPC}
\newcommand{\E}{\mathcal E}
%

%[[PK environments for article style
\newtheorem{theorem}{Theorem}
\newtheorem{conjecture}[theorem]{Conjecture}
\newtheorem{corollary}[theorem]{Corollary}
\newtheorem{definition}[theorem]{Definition}
\newtheorem{lemma}[theorem]{Lemma}
\newtheorem{observation}[theorem]{Observation}
\newenvironment{proof}[1][Proof]{\noindent\textbf{#1.} }{\hfill $\Box$\\[2mm]}
\newenvironment{proofsketch}[1][Proof sketch]{\noindent\textbf{#1.} }{\hfill $\Box$\\[2mm]}
%]]

\begin{document}
\sloppy

%\title{Distributed Network Programming}
%\title{The Distributed SDN Update Problem:\\Towards Software Transactional Networks}

%\title{On Consistent Updates in Software Defined Networks under Unreliable Control}
%\title{The Case for Reliable Software Transactional Networking}

%\title{A Distributed SDN Control Plane for Concurrent Policy Updates}

\title{Distributed Software-Defined Networking:\\ ACM PODC 2014 Workshop }



\author{
Petr Kuznetsov$^{1}$ \quad Stefan Schmid$^{2}$\\
\\
       $^{1}$ T\'el\'ecom ParisTech\\
%, 46 Rue Barrault, 75013 Paris, France\\
        petr.kuznetsov@telecom-paristech.fr\\
\\
        $^{2}$ TU Berlin \& T-Labs \\ %Ernst-Reuter Platz 7, 10587 Berlin, Germany\\
	    stefan.schmid@tu-berlin.de}
%	}

%\institute{}

\date{}


\maketitle


\thispagestyle{empty}

%\if \SAVESPACE 1
%\setlength{\floatsep}{3pt}
%\setlength{\textfloatsep}{3pt}
%\setlength{\dbltextfloatsep}{3pt}
%\setlength{\intextsep}{3pt}
%\setlength{\abovecaptionskip}{3pt}
%\fi

% A category with the (minimum) three required fields
%\category{C.2.1}{Network Architecture and Design}{Centralized Networks}
%\category{C.2.4}{Distributed Systems}{Network Operating Systems}
%\terms{Measurement, Performance}
%\keywords{}



\begin{abstract}
The first workshop on Distributed Software-Defined Networking took
place in Paris, France, on the 15th of July, just before 
the 33rd ACM Symposium on Principles of Distributed Computing. 
The workshop intended to be a forum to discuss new algorithmic and
distributed computing challenges offered by the emerging field of
Software Defined Networking (SDN).
For the workshop, we invited invited researchers in the fields of
distributed computing and networking in order to understand whether distributed implementations of the
SDN control plane give rise to new and interesting research questions,
in solving which  the expertise of the PODC community may be of use.
\end{abstract}


\section{SDN and PODC?}

Computer networking currently goes through a phase transition, and the paradigm of Software-Defined Networking (SDN)
is discussed intensively, both in the industry and in the academia. In
a nutshell, SDN out-sources the control over the network 
to a logically centralized software, called the \emph{control plane}. 
The ability of the control plane to ``program'' the network (the
\emph{data plane}) opens new interesting opportunities.

 While the perspective of a centralized controller simplifies network management,
it comes with the usual drawbacks: single-point of failure,
scalability bottleneck, overhead due to indirection, etc. 
This raises the questions: Do we need a \emph{distributed} control plane for SDN? And if so,
what could the ``PODC community'' contribute?

In order to discuss these questions, we organized the workshop on Distributed
Software-Defined Networks (DSDN). The workshop took
place in Paris on the sunny day of July 15th, just before
PODC, concurrently with four other PODC workshops, and gathered the
audience of about 20-40 people (which could be considered as success). 

The program consisted of invited and peer-reviewed presentations from researchers working in applied networking,
systems, and theory of distributed computing.


\section{Program}

Below we give a short overview of the talks given in our
workshop. Abstracts and slides can be found at \url{http://www.podc.org/podc2014/dsdn14/}. 

\subsection{Foundations of SDN}

In the opening keynote, Nate Foster (Cornell) gave an overview of
programming abstractions for SDN. The talk consisted of three parts. 

... 


\subsection{Consistent Range Classification with OpenFlow} 

Yehuda Afek (TAU) presented his recent work with Anat Bremler-Barr and
Liron Schiff~\cite{AfekBS14}.   

...

\subsection{SDN-Based Private Interconnection}

...

\subsection{Software Transactional Networking: A Robust and
  Distributed SDN Control Plane}

Marco Canini (UCL) presented the concept of software transactional
networking (STN), a control-plane abstraction used for consistent
composition of concurrent policies~\cite{stn,tr-stn}. The abstraction assumes a set of
control applications that concurrently apply modifications (or
updates) of the
\emph{network policy}, i.e., the set of rules that stipulate how the
traffic should be processed at the data plane.     
Given that the policy updates coming from different control applications may
conflict with each other, the STN framework offers to the applications
a \emph{transactional} interface with \emph{all-or-nothing} semantics.   
The talk hinted on the formal definition of the abstraction of consistent policy
composition (CPC), and sketched designs of its resilient implementations.  
 
\subsection{Declarative, Distributed Configuration}

Can the challenges motivating the use of SDN be addressed using
existing hardware and protocols. Sanjai Narain (Applied Communication Sciences) summarized the
experience collected at his company in using the Assured and Dynamic
Configuration(ADC) system. In their approach, network functionality is
expressed as a set of constraints on configuration variables. SAT or
SMT solvers are used to convert these constraints into values of configuration variables.
Of course, in specific scenarios, proprietary solutions may be more
efficient and  easier to deploy than generic ones (e.g., based on the
SDN framework), and Sanjai's talk questioned the very
motivation behind migrating to SDN. 

\subsection{Managing the Network with Merlin}

...

\subsection{Managing Dynamic Networks: Distributed or Centralized
  Control}

In his somewhat provocative talk, Roger Wattenhofer (ETHZ) considers
the following question. If we take the extreme case of a network 
managed (e.g., using the SDN approach) by a (fault-tolerant,
performant, etc.) central controller, would this imply
that distributed algorithms are no longer needed. The answer (not very
surprisingly) is no: we still have to deal with the problem of
dynamicity and failures on the data plane, as well as the fact that
the data plane is inherently geographically distributed and cannot be
manipulated in the atomic manner. In short, we still have to deal with
\emph{consistency} of network control, and here we can benefit from
distributed computing which is essentially all about consistency. 
In his talk, Roger overviewed several natural network consistency
criteria (such as \emph{loop-freedom} or \emph{per-packet
  consistency}) and sketched several impossibility results and
complexity bounds of achieving these criteria in a few different
network models.   


\section{Distributed SDN: New? Interesting?}

A recently published detailed survey on the work that has been in
SDN-related research so far~\cite{sdn-survey} 
claims that ``the myth that logical centralization implied
a physically centralized controller'' has been addressed and, as a
result, ``SDN ideas have matured and evolved from
an academic exercise to a commercial success''. 
Interestingly, among 407 articles cited in the survey, we do not find
a single paper that appeared in a distributed-computing theory venue
(such as PODC or DISC).
We think, based on our personal research experience, this is not
because SDN does not give rise new and interesting
research questions related to distributed computing. 
Maybe it is just lack of curiosity or insights on recent
advances in networking?  
 
The workshop shows that the situation is slowly changing: a few
research groups are looking at the distributed aspects of SDN now.    
It seems, however, that progress here is still rather modest and
typically boils down to solving ``conventional'' algorithmic problems, 
using SDN only as a motivating application. We believe nevertheless that
there are deeper distributed challenges coming directly from SDN.  


\bibliographystyle{abbrv}
\bibliography{references}  % main.bib is the name of the Bibliography in this case

\end{document}
