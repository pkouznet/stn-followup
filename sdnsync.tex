\documentclass[11pt,pdftex,letter]{article}
%\documentclass[11pt]{llncs}
%\documentclass[11pt,pdftex]{article}
%\documentclass{sig-alternate-10pt}
%\usepackage{amsthm}
%\usepackage{algorithm}
%\usepackage[noend]{algorithmic}
\usepackage[lined,boxed,commentsnumbered]{algorithm2e}


\usepackage{amssymb}
\usepackage{comment}
\usepackage{amsmath}
%\usepackage{graphicx}
%\usepackage{color}
\usepackage{fullpage}
%\usepackage[pdftex]{graphicx}
%\DeclareGraphicsRule{*}{mps}{*}{<++>}

\ifx\pdftexversion\undefined
\usepackage[dvips]{graphicx}
\else
  \usepackage[pdftex]{graphicx}
  \DeclareGraphicsRule{*}{mps}{*}{}
\fi


\def\lf{\tiny}
\def\rrnnll{\setcounter{linenumber}{0}}
\def\nnll{\refstepcounter{linenumber}\lf\thelinenumber}
\newcounter{linenumber}

\usepackage[usenames,dvipsnames]{color}
%\usepackage[caption=true,font=footnotesize]{subfig}
\usepackage{subfigure}
%\usepackage{xspace} 	% Guesses whether a space is needed when invoked
\usepackage{listings}
\usepackage{url}
\usepackage{wrapfig}
\usepackage{cite}
%\usepackage{framed}
\usepackage{framed,color}
\definecolor{shadecolor}{rgb}{0.9,0.9,0.9}
%
\usepackage[boxed]{algorithm2e}
%\usepackage{comment}

%[[PKto change spacing
%\usepackage{titlesec}
%\titlespacing\section{0pt}{7pt}{6pt}
%]]

%\newtheorem{theorem}{Theorem}[section]
%\newtheorem{takeaway}[theorem]{Takeaway}
%\newtheorem{fact}[theorem]{Fact}


%\pdfpagewidth=8.5in
%\pdfpageheight=11in

% url.sty was written by Donald Arseneau. It provides better support for
% handling and breaking URLs. url.sty is already installed on most LaTeX
% systems. The latest version can be obtained at:
% http://www.ctan.org/tex-archive/macros/latex/contrib/misc/
% Read the url.sty source comments for usage information. Basically,
% \url{my_url_here}.

\newcommand{\CPO}{\textsc{FixTag}}
\newcommand{\DPO}{\textsc{ReuseTag}}
\newcommand{\PS}{\textsc{PS}}
\newcommand{\Bit}{\textsc{Bit}}

\newcommand{\gns}{Global Network State\xspace}
\newcommand{\GNS}{\textsc{GNS}\xspace}

%\newtheorem{theorem}{Theorem}[section]
%\newtheorem{lemma}{Lemma}[section]
%\newtheorem{claim}{Claim}[section]
%\newtheorem{scenarios}{Scenarios}[section]
%\newtheorem{observation}{Observation}
%\newtheorem{takeaway}{Takeaway}[section]
%\newtheorem{definition}{Definition}


%
%
% carry over Herald's group cool way of marking changes
\definecolor{heraldBlue}{rgb}{0.0,0.0,0.8}
\definecolor{heraldRed}{rgb}{0.8,0.0,0.0}
\definecolor{heraldGray}{rgb}{0.4,0.4,0.4}
\definecolor{heraldBlack}{rgb}{0.0,0.0,0.0} %removes comment color
\definecolor{heraldGreen}{rgb}{0.0,0.4,0.0} %removes comment color
\def\r#1{\textcolor{heraldBlue}{\em #1}}
\def\q#1{\textcolor{heraldRed}{\em #1}}
\def\d#1{\textcolor{heraldBlue}{#1}}
\def\R#1{\textcolor{heraldBlue}{#1}}
\def\D#1{\textcolor{heraldBlue}{#1}}
%
%\DeclareMathOperator{\respc}{res}

\newcommand{\len}{\text{d}}

\newcommand{\nodes}{\mathcal{N}}
\newcommand{\links}{\mathcal{E}}
\newcommand{\epoints}{\Pi}
\newcommand{\epoint}{\pi}
\newcommand{\legsw}{\mathcal{L}}
\newcommand{\sdnsw}{\mathcal{S}}
\newcommand{\sw}{\legsw\sqcup\sdnsw}
\newcommand{\fset}{FS}

\newcommand{\Cost}{\gamma}

\newcommand{\ft}{ft}

\newcommand{\eepath}{p}
\newcommand{\link}{e}

\newcommand{\dom}{\textit{dom}}
\newcommand{\pr}{\textit{pr}}
\newcommand{\CPOs}{\textit{paths}}
\newcommand{\seq}{\textit{seq}}
\newcommand{\cur}{\textit{cur}}
\newcommand{\Tag}{\textit{tag}}


\newcommand{\ports}{\Pi}
\newcommand{\seport}{\pi}
\newcommand{\inports}{\overrightarrow{\ports}}
\newcommand{\swports}{\overleftrightarrow{\ports}}
\newcommand{\seinports}{\epoints^{\bullet}}
\newcommand{\seinportsone}{\seinports_1}
\newcommand{\seinportstwo}{\seinports_2}
\newcommand{\seinportsthree}{\seinports_3}
\newcommand{\neinports}{\epoints^{\circ}}

\newcommand{\readt}{\texttt{r}}
\newcommand{\writet}{\texttt{w}}
\newcommand{\op}{\texttt{op}}


\newcommand{\cellblocks}{CB}
\newcommand{\cellblock}{c}
\newcommand{\frontier}{\mathcal{F}}
\newcommand{\MIP}{\textsc{Opt}}
\newcommand{\smartparagraph}[1]{\noindent{\bf #1}\ }
\newcommand{\eg}{{\it e.g.}}
\newcommand{\ie}{{\it i.e.}}
\newcommand{\etc}{{\it etc.}}
\newcommand{\etal}{{\it et al.}\xspace}
\newcommand{\id}{{\it id}}

\def\TR{0}
\def\NOTES{1}
\def\SAVESPACE{1}
\def\SHOWAUTHORS{1}
\def\SHOWGIT{0}
% variables may contain the above definitions to control build options at compile-time
%\input{variables}

\if \SAVESPACE 1
%\usepackage{setspace}
%\usepackage{titlesec}
%\titlespacing\section{0pt}{7pt}{6pt}
\usepackage{titlesec}
\titlespacing\section{0pt}{7pt}{6pt}

\fi

\if \NOTES 1
\newcommand{\mcnote}[1]{\textcolor{heraldBlue}{\small \bf [MC: #1]}}
\newcommand{\dlnote}[1]{\textcolor{heraldBlue}{\small \bf [DL: #1]}}
\newcommand{\ssnote}[1]{\textcolor{heraldBlue}{\small \bf [SS: #1]}}
\newcommand{\pknote}[1]{\textcolor{heraldBlue}{\small \bf [PK: #1]}}
%\newcommand{\problem}[1]{\textcolor{heraldRed}{\small \bf [PROBLEM: #1]}}
\else
\newcommand{\mcnote}[1]{}
\newcommand{\dlnote}[1]{}
\newcommand{\ssnote}[1]{}
\newcommand{\pknote}[1]{}
\newcommand{\problem}[1]{}
\fi

%Petr's definitions
\newcommand{\ack}{\textit{ack}}
\newcommand{\nack}{\textit{nack}}
\newcommand{\rmw}{\textit{rmw}}
\newcommand{\State}{\mathit{States}}
\newcommand{\ignore}[1]{}
\newcommand{\Pb}{CPC}
\newcommand{\E}{\mathcal E}
%

%[[PK environments for article style
\newtheorem{theorem}{Theorem}
\newtheorem{conjecture}[theorem]{Conjecture}
\newtheorem{corollary}[theorem]{Corollary}
\newtheorem{definition}[theorem]{Definition}
\newtheorem{lemma}[theorem]{Lemma}
\newtheorem{observation}[theorem]{Observation}
\newenvironment{proof}[1][Proof]{\noindent\textbf{#1.} }{\hfill $\Box$\\[2mm]}
\newenvironment{proofsketch}[1][Proof sketch]{\noindent\textbf{#1.} }{\hfill $\Box$\\[2mm]}
%]]

\begin{document}
\sloppy

%\title{Distributed Network Programming}
%\title{The Distributed SDN Update Problem:\\Towards Software Transactional Networks}

%\title{On Consistent Updates in Software Defined Networks under Unreliable Control}
%\title{The Case for Reliable Software Transactional Networking}

%\title{A Distributed SDN Control Plane for Concurrent Policy Updates}

\title{Programming SDN: A Transactional Approach}



\author{
	Marco Canini$^{1}$ \quad Petr Kuznetsov$^{2}$ \quad Dan
        Levin$^{3}$ \quad Stefan Schmid$^{4}$\\
%\thanks{Contact author:
 %         stefan@net.t-labs.tu-berlin.de, FG INET, TEL 16,
  %        Ernst-Reuter Platz 7, 10587 Berlin, Germany}
\\
        $^{1}$ Universit\'{e} catholique de Louvain, Place Sainte Barbe 2, 1348 Louvain-la-Neuve, Belgium\\
        marco.canini@uclouvain.be\\
\\
        $^{2}$ T\'el\'ecom ParisTech, 46 Rue Barrault, 75013 Paris, France\\
        petr.kuznetsov@telecom-paristech.fr\\
\\
        $^{3}$ TU Berlin, Marchstr. 23, 10587 Berlin, Germany\\
        dlevin@inet.tu-berlin.de\\
\\
        $^{4}$ TU Berlin \& T-Labs, Ernst-Reuter Platz 7, 10587 Berlin, Germany\\
	    stefan.schmid@tu-berlin.de}
%	}

%\institute{}

\date{}


\maketitle


\thispagestyle{empty}

%\if \SAVESPACE 1
%\setlength{\floatsep}{3pt}
%\setlength{\textfloatsep}{3pt}
%\setlength{\dbltextfloatsep}{3pt}
%\setlength{\intextsep}{3pt}
%\setlength{\abovecaptionskip}{3pt}
%\fi

% A category with the (minimum) three required fields
%\category{C.2.1}{Network Architecture and Design}{Centralized Networks}
%\category{C.2.4}{Distributed Systems}{Network Operating Systems}
%\terms{Measurement, Performance}
%\keywords{}



\begin{abstract}
Computer networking currently goes through a transition phase:
the paradigm of Software-Defined Networking (SDN)
is discussed intensively, both in the industry and in the academia. In
a nutshell, SDN out-sources the control over the network 
to a logically centralized software, called the \emph{control plane}. 
The ability of the control plane to ``program'' the network (the
\emph{data plane}) opens new interesting opportunities to network
operators and system designers.
In is important that the execution of such programs should appear
transparent to the network. 
Intuitively, no user application should be negatively affected by the
concurrent configuration changes.
We propose a way to capture this intuition formally and describe
algorithms to implement it.    
\end{abstract}

%\vspace{1cm}

%\begin{center}
%{\bf [Regular paper only]}
%\end{center}
%[[PK I guess not anymore?]]

%\vspace{1cm}

%\begin{center}
%{\emph{Contact Address:}\\Marco Canini, Place Sainte Barbe~2, 1348 Louvain-la-Neuve, Belgium\\Tel: $+$32 10 47 48 32,
%marco.canini@uclouvain.be}
% {\emph{Contact Address:}\\Stefan Schmid, MAR 4-4, Marchstr.~23, 10587 Berlin, Germany\\Tel: $+$49 175 930 98 75,
% stefan.schmid@tu-berlin.de}
%\end{center}


%\newpage

%\begin{center}
%{\bf Regular and student paper: Dan Levin is a full-time student.}
%\end{center}

\newpage
\pagenumbering{arabic}\setcounter{page}{1}

%\section*{todo before submission}

\begin{comment}


\section{Introduction}




\bibliographystyle{abbrv}
\bibliography{references}  % main.bib is the name of the Bibliography in this case


\end{document}
